\documentclass[10pt,a4paper]{article}

\usepackage{kerkis}
\usepackage[T1]{fontenc}
\usepackage{alphalph}
\usepackage[utf8x]{inputenc}
\usepackage{pgfplots}
\pgfplotsset{compat=1.18, width=10cm}

\usepackage{ucs}
\usepackage[english]{babel}
\usepackage{listings}
\usepackage{fancyhdr}
\usepackage{graphicx}
\usepackage{geometry}
\usepackage{wrapfig}
\usepackage{caption}
\usepackage{float}
\usepackage{enumitem}
\usepackage{bm}
\usepackage{amsmath}
\usepackage{algorithm}
\usepackage{algpseudocode}
\usepackage{color}
\usepackage{url}
\usepackage{amssymb}
\usepackage{accents}
\usepackage{xfrac}
\geometry{margin=2cm}
\usepackage{multirow}
\usepackage{caption}
\usepackage{hyperref}
\usepgfplotslibrary{fillbetween}
\captionsetup[table]{position=bottom} 
\hypersetup{
	colorlinks=true,
	linkcolor=blue,
	filecolor=magenta,      
	urlcolor=blue}


\pagestyle{fancy}
\title{2nd Homework Assignment \\ \huge{Project on Support Vector Machines}}
\author{Vasileios Papageorgiou}
\date{\today}
\fancyhead[L]{Optimization Techniques}
\fancyhead[R]{MSc Program in Data Science (PT)}

\newcounter{para}
\newcommand\mypara{\par\refstepcounter{para}\textbf{\thepara.}\space}
\setlength\parindent{0pt}

\renewcommand{\thesubsection}{(\alphalph{\value{subsection}})}

\begin{document}
	\maketitle
	\thispagestyle{fancy}
	
	

\section*{Theoritical Background}

We have the following non linear program:

\begin{equation}\label{eq:3}
	\begin{aligned}
		\min \{ F(x) = \frac{c^T x}{d^T x} : A x = b; \, x \geq 0 \}
	\end{aligned}
\end{equation}



\begin{algorithm}
	\caption{Bisection Method for Optimal $\lambda$}
	\begin{algorithmic}[1]
		\State \textbf{Given:} interval $[L, U]$ that contains optimal $\lambda$
		\Repeat
		\State $\lambda := \frac{u + l}{2}$
		\State Solve the feasibility problem:
		\State $\quad c^T x \leq \lambda d^T x $
		\State $\quad d^T x > 0$
		\State $\quad Ax = b$
		\State Adjust the bounds
		\If{feasible}
		\State $U := \lambda$
		\Else
		\State $L := \lambda$
		\EndIf
		\Until{$U - L \leq \epsilon$}
	\end{algorithmic}
\end{algorithm}

\section*{Problem 4}
 

\end{document}